\documentclass[12pt]{article}

    \usepackage[dvipsnames,svgnames,x11names]{xcolor}
    \usepackage{graphicx}
    \usepackage{tikz}
    \usepackage{amsmath}
    \usepackage[fleqn,tbtags]{mathtools}
    \usepackage{physics}
    \usepackage{array}
    \usepackage{geometry}
	\usepackage{hyperref}
	\usepackage{multirow}
	\usepackage{makecell}
	
    
    \geometry{
        top=1.5in,
        left=1in,
        right=1in,
        bottom=1.5in
    }

    % Definir el ancho total disponible
    \newlength{\totalwidth}
    \setlength{\totalwidth}{\textwidth}

    % Calcular anchos de columna para la tabla principal
    \newlength{\firstcolwidth}
    \setlength{\firstcolwidth}{0.25\totalwidth}
    \newlength{\othercolwidth}
    \setlength{\othercolwidth}{0.25\totalwidth}

    % Comandos básicos
    \newcommand{\ans}[1]{\textbf{\color{black} #1}}
    \renewcommand{\vec}[1]{\boldsymbol{#1}}
    \newcommand{\unitvec}[1]{\boldsymbol{\hat{#1}}}

    % Tipos de columna personalizados
    \newcolumntype{Y}{>{\centering\arraybackslash}p{\othercolwidth}}
    \newcolumntype{Z}{>{\centering\arraybackslash}p{\firstcolwidth}}

\begin{document}
    \title{Comparación de los métodos de selección}
    \author{The Bankers}
    \date{Noviembre 2024}
    \maketitle
	\section{Implementación}
		Visita \href{https://www.overleaf.com}{Github} Código fuente del informe.
	
	\section{Resultados}
		\subsection[Experimento 1]{Experimento 1: Comparación de métodos de selección}
			\subsubsection[Roulette Wheel Selection]{Roulette Wheel Selection}
				Consiste en que la probabilidad de ser elegido es proporcional al fitness y cada elemento ocupa un segmento para ser elegido.
			
			\subsubsection[Rank Based Selection]{Rank Based Selection}
				La probabilidad está en función al rango o posición de los elementos de una población, cada elemento se ordena según el fitness.
			
			\subsubsection[Fitness Scaling Selection]{Fitness Scaling Selection}
				El fitness de cada elemento se realiza un escaldo lineal o adaptativo antes del proceso de selección.

			\subsubsection[Tournament Selection]{Tournament Selection}
				Este método consiste en elegir de forma aleatoria elementos de una población quedando solo con el óptimo.

			\newpage
			{\large Cuadro comparativo}
			\vspace{3mm}
			\begin{center}
				\begin{tabular}{|>{\centering\arraybackslash}p{0.15\textwidth}|*{4}{>{\centering\arraybackslash}p{0.2\textwidth}|}}
					\hline
					\multicolumn{1}{|c|}{\multirow{2}{*}{\rule{0pt}{4ex}Generación}} & \multicolumn{4}{c|}{Fitness} \\[2ex]
					\cline{2-5}
					& \rule{0pt}{3ex}\makecell{Roulette\\Wheel\\Selection} & \makecell{Rank-Based\\Selection} & \makecell{Fitness\\Scaling\\Selection} & \makecell{Tournament\\Selection} \\[1.5ex]
					\hline
					\rule{0pt}{4ex}0 & 46.0405753 & 45.372529 & 47.1906354 & 45.5480999 \\[1ex]
					\hline
					\rule{0pt}{4ex}200 & 21.7275399 & 45.372529 & 27.3317811 & 16.2008103 \\[1ex]
					\hline
					\rule{0pt}{4ex}400 & 17.4071704 & 45.372529 & 23.3323755 & 12.0895081 \\[1ex]
					\hline
					\rule{0pt}{4ex}600 & 14.2522011 & 44.1501087 & 19.9919763 & 10.372674 \\[1ex]
					\hline
					\rule{0pt}{4ex}800 & 12.8809428 & 43.9863321 & 18.7416694 & 9.90466136 \\[1ex]
					\hline
					\rule{0pt}{4ex}1000 & 12.6590962 & 42.7031294 & 16.6458018 & 9.90466136 \\[1ex]
					\hline
					\rule{0pt}{4ex}1200 & 12.532009 & 42.7031294 & 15.43604 & 9.8294855 \\[1ex]
					\hline
					\rule{0pt}{4ex}1400 & 12.4233396 & 42.7031294 & 14.9056858 & 9.74728195 \\[1ex]
					\hline
					\rule{0pt}{4ex}1600 & 12.1921104 & 42.7031294 & 14.3798882 & 9.59585609 \\[1ex]
					\hline
					\rule{0pt}{4ex}1800 & 11.9874773 & 42.7031294 & 13.6739941 & 9.12644879 \\[1ex]
					\hline
					\rule{0pt}{4ex}2000 & 11.1098868 & 42.3230657 & 12.6565647 & 9.11431122 \\[1ex]
					\hline
					\rule{0pt}{4ex}2200 & 10.7993498 & 41.8715207 & 11.9694388 & 8.93188524 \\[1ex]
					\hline
					\rule{0pt}{4ex}2400 & 10.5744095 & 41.5596853 & 11.814441 & 8.92229561 \\[1ex]
					\hline
					\rule{0pt}{4ex}2600 & 10.5310081 & 40.4733506 & 11.8080908 & 8.92229561 \\[1ex]
					\hline
					\rule{0pt}{4ex}2800 & 10.4720815 & 40.4733506 & 11.8080908 & 8.8645615 \\[1ex]
					\hline
				\end{tabular}
			\end{center}
		\subsection[Experimento 2]{Experimento 2: Comparación de métodos de inicialización de población}
			El segundo experimento consiste en la ejecucion del algoritmo genetico con una poblacion inicial aleatoria y heuristica, para este caso se utilizara Hill Climbing.

	\section{Discución}

    % {\large Unknown state $\ket{\psi_1}$}
    % \vspace{3mm}

    % \begin{center}
    %     \begin{tabular}{|Z|Y|Y|Y|}
    %         \hline
    %         \rule{0pt}{4ex}Probabilities & \multicolumn{3}{c|}{Axis} \\[2ex]
    %         \hline
    %         \rule{0pt}{4ex}Result & $x$ & $y$ & $z$ \\[2ex]
    %         \hline
    %         \rule{0pt}{4ex}$S_i=\hbar$ &&& \\[2ex]
    %         \hline
    %         \rule{0pt}{4ex}$S_i = 0$ &&& \\[2ex]
    %         \hline
    %         \rule{0pt}{4ex}$S_i = - \hbar$ &&& \\[2ex]
    %         \hline
    %     \end{tabular}
    % \end{center}

    % \vspace{1cm}
    % {\large Unknown state $\ket{\psi_2}$}
    % \vspace{3mm}

    % \begin{center}
    %     \begin{tabular}{|Z|Y|Y|Y|}
    %         \hline
    %         \rule{0pt}{4ex}Probabilities & \multicolumn{3}{c|}{Axis} \\[2ex]
    %         \hline
    %         \rule{0pt}{4ex}Result & $x$ & $y$ & $z$ \\[2ex]
    %         \hline
    %         \rule{0pt}{4ex}$S_i=\hbar$ &&& \\[2ex]
    %         \hline
    %         \rule{0pt}{4ex}$S_i = 0$ &&& \\[2ex]
    %         \hline
    %         \rule{0pt}{4ex}$S_i = - \hbar$ &&& \\[2ex]
    %         \hline
    %     \end{tabular}
    % \end{center}

    % \newpage
    % {\large Spin 1 Interferometer}
    % \vspace{3mm}
	\renewcommand{\contentsname}{Índice}  % Cambiar título
	\tableofcontents
	\setcounter{tocdepth}{3} 

\end{document}